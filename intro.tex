\section{\label{sec:level1}Introduction} The aim of cryptography is to provide mathematical tools to keep the contents of private messages indecipherable from a third party observer as the message is being transferred between a source and a receiver \citep{ShenoyHejamadi2017QuantumBeyond}. Cryptography is of importance in sectors such as the ministry of defense and the financial trading market \citep{Diamanti2016PracticalDistribution}. Additionally, the popularity of online payments and banking, including the recent surge in popularly of cryptocurrency, risks personal sensitive data and assets if the security protocols are not in place or become compromised. In 1994 due to the development of quantum algorithms, such as Shor's algorithm and the possibility of building a quantum computer provided evidence that cryptography schemes in the future could become vulnerable to a new type of crytanalysis. The risk to classical cryptography protocols is due the potential for computation of discrete logarithms and factorising integers \citep{ShorAlgorithmsFactoring}. However, the concept of quantum key distribution (QKD) was a very early development in the field of quantum computation, therefore in 1984 the BB84 protocol was theorised \citep{unknown, Bennett2014QuantumTossing}. By 1992 the BB84 scheme was experimentally realised \citep{Bennett1992ExperimentalCryptography}.

The BB84 protocol was the first in a series of devised protocols, discussed in Ref. [\citen{Singh2014QuantumReview}], which rely on the Heisenberg uncertainty to prevent a third party gaining information without being detected. Furthermore, another series of protocols utilised quantum entanglement, with the most famous example being the Erkert protocol \citep{Singh2014QuantumReview}. However, there are several challenges facing the implementation of a large-scale QKD network including; long distance transmission of keys being limited by transmission losses, the requirement for single photon sources and detectors, the need for rapid quantum random generators and protection against attack protocols \citep{Diamanti2016PracticalDistribution,Bedington2017ProgressDistribution}.



   