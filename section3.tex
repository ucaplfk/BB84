\section{\label{sec:level1}Discussion and Conclusion}
The significant difference between the estimated laser pulse width and quoted pulse width is suspected, at least in part, to be due to broadening of the FWHM due to the ultimate time resolution electronic detection being 0.21 ns and APD jitter time. The jitter time is the statistical fluctuation of the photon arrival time at the detector and the output of the electrical pulse \citep{Jani2007TimingCircuitry}. The jitter pulse is < 60 ps and $\approx$ 350 ps for the two APD types. The estimated 0.06 photons per pulse appears reasonable as BB84 schemes have been known to often be operated at $\bar{N}\approx0.1$ \citep{Fox2006QuantumIntroduction}. The dark count rate and the inability for the APD to complete two photon detection increases the uncertainty of the experimentally measured $\bar{N}$. The APD known dark count rates are lower than the experimentally measured values. This is not surprising due to the approximation used to gather dark count data. Additionally, if the fiber-coupling efficiency is not optimised the difference may also be due to photon state losses. 

Following the correction for differences in the APD efficiencies, the state detection results are shown in Fig. (\ref{fig:countsvspolarisationstate}). The expected characteristic results of the BB84 protocol for each generated EOM photon state is observed. Measurement in the correct basis results in the largest normalised count, whilst the approximately equal bars correspond photons being reflected to the opposite basis measurement arm. The QBER calculated determines that the scheme is not currently sufficiently secured. This is due to two of the error rates for two of the prepared states being about the 15$\%$ cut-off for completing secured quantum cryptography. Factors contributing to the increased QBER include dark counts and background light APD detection, imperfections of the EOM operation which can result in rotation of the plane of polarisation. The amplifier places limitations on the experiment where the trigger repetition rate determined from Fig. (\ref{fig:triggerevents}) is given as 1 MHz. The trigger rate is limited by the switching rate of the EOM amplifier. Additionally the voltage range being between $-V_{\pi}$ and $+V_{\pi}$, reduces the ability to generate light with only a polarisation component in the $\ket{H}$ direction. 

In conclusion, analysis of the data collected by a previous group enables useful insight into the current limiting factors of the experimental set-up. By addressing aspects such as improving the fiber the coupling efficiency, further shielding of the set-up from background light sources and optimising the voltage at which EOM produces the four light polarisation states could reduce the noise sources in the free-space optical set-up. Therefore, reducing the noise level of the imperfect quantum channel such that the QBER for the detection of each photon state is below < 15 $\%$ would allow the scheme to securely distribute a one-time pad.



