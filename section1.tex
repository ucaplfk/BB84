\section{\label{sec:level1}QKD Experimental Implementation}
Quantum cryptography enables a method to share a private key between a source (Alice) and a detector (Bob) while being able to detect the presence of a third-party observer (Eve) who attempts to gain information about the key. The shared key provides a one-time pad which enables Alice to encrypt a message the same length as the key string by adding the corresponding key and message bits together. Bob receives the encrypted message via a public channel and he obtains the message by subtracting the key \citep{Nielsen2010QuantumInformation}.


\subsection{The BB84 protocol}
In the BB84 protocol Alice initially produces a random bit string of length $m$. Alice has the ability to prepare photon states which can be in the horizontally/vertically polarised ($\ket{H}$, $\ket{V}$) basis or right/left-circularly polarised ($\ket{R}$, $\ket{L}$) basis. The states in each basis are allocated as a '0' or '1' bit value eg. $\ket{H}$ = '0' and $\ket{V}$ = '1'. Therefore based on the value of each bit of the randomly generated string, the photon state is chosen randomly between the two polarisation states from different basis which have the correct assigned bit value. Alice sends the $m$ photon states to Bob. Bob randomly decides which basis to measure the prepared states where consequently the states are projected to classical bits. For $\approx$ 50$\%$ of Bob's measurements are the same basis as Alice has prepared the state  and Bob will obtain the correct value of the bit. 50$\%$ of the time Bob's result is completely random. Alice and Bob publicly communicate prepared and measured basis without exposing the values of the bit string or prepared photon states. They retain the bits in which the basis are in agreement. Now the shared key bit string length is reduced to $m/2$ \citep{Fox2006QuantumIntroduction,Bedington2017ProgressDistribution}. 

Following from the no-cloning theorem \citep{Wootters1982ACloned}, Alice and Bob can determine the presence of a third party observer. If Eve captures an encoded photon state, she will make a measurement in a random basis. Thereafter Eve prepares photon in the same basis corresponding to the classical bit result and sends it to Bob. If she guesses the correct basis she gains information of the state and transmits the correct state to Bob. Otherwise there is a 50$\%$ probability of obtaining a random bit value which is correct even when she measures in the incorrect basis. Once the key has been transferred from Alice to Bob, Bob then sends a selection, often $m/4$ of his bits to Alice via the unsecured classical channel. Alice compares her bits to the selection Bob has sent her and then she discards of these bits. The length of the remaining key string is now $m$/4. Under the assumption that the quantum channel is noiseless, if the quantum bit error rate (QBER) is >25$\%$ then Alice can determine that Eve has been eavesdropping and the key is discarded. If the QBER is <25$\%$ the communication of the key is deemed secure \citep{Fox2006QuantumIntroduction,Bedington2017ProgressDistribution}. Error correction is applied to the imperfect key to correct for differences between the key bit strings. This process is known as information reconciliation. Additionally, privacy amplification is used to minimise Eve's knowledge of the distilled key following error-correction \citep{Nielsen2010QuantumInformation}. 


%(\ref{fig:})%
\subsection{Free-space Implementation} 
The set-up must have the capability to produce the four photon states ($\ket{H}$, $\ket{V}$, $\ket{R}$, $\ket{L}$) and have a specific detector assigned for each state. To send the photon states a single photon source is required. The production of single photons is an area of active research, with quantum dot devices currently leading in the race to produce an on demand source of anti-bunched light \citep{Senellart2017High-performanceSources}. The 635 nm diode laser used in this implementation produces a coherent quantum state. Therefore pulsing at <100 MHz and attenuating the power such that the there is a small mean number of photons per pulse, $\bar{N}\approx0.1$, results in $\approx 5\%$ of the pulses containing more than 1 photon \citep{Fox2006QuantumIntroduction}. 

The optical set-up is shown in Fig. (\ref{fig:opticalsetup}) where the light is coupled into the single-mode optical fibre core using an arrangement of focusing lenses. The use of pairs of mirrors increase the degrees of freedom to assist with beam alignment. Initially considering Alice's optical set-up, the optical isolator consists of the Faraday rotator sandwiched between two polarisers. The linearly polarised light is rotated by the applied magnetic flux density of the Faraday rotator such that light is transmitted but cannot be reflected back to the laser source \citep{Saleh2007FundamentalsPhotonics}. The noise eater contains a proportional-differential-integral controller circuit which stabilises the laser intensity. Electro-Optic Modulator (EOM) produces an electronically controlled variable waveplate due to the change in the refractive index, $n$, of the anisotropic crystal in the presence of an applied electric field, $E$. The light undergoes a phase shift given by:

\begin{equation}
\label{eq:interactionham0}
\Delta\phi = -\pi \frac{\chi n^{3}EL}{\lambda },
\end{equation}

where $\lambda$ is the laser wavelength, $L$ is the crystal length and $\chi$ is the electric susceptibility. The distance, $d$ separates the contacts of the voltage applied on different faces of the crystal. To apply a phase shift of $\pi$, a voltage $V_{\pi}=(d\lambda)/(L\chi n^{3})$ is required \citep{Saleh2007FundamentalsPhotonics,Iizuka2002ElementsMedia}. To prepare all four polarisation states the required voltage range is $-V_{\pi}$ to $+V_{\pi}$. 

The voltage signals applied to implement the QKD scheme when used in practice to encode private messages should be generated using a random number generator. However the photon states for this experiment are generated using computational control of a signal generator to produce a repeating sequence of small voltage signals which are sent to the amplifier. Therefore considering that the incoming light is always in the $\ket{V}$ state and for an amplifier voltage sequence : $0, -V_{\pi},-\frac{1}{2}V_{\pi}, \frac{1}{2}V_{\pi}$ the states: $\ket{V}$, $\ket{H}$, $\ket{R}$, $\ket{L}$ are prepared. Since the amplifier used for this experiment reaches half the required voltage range, the beam must be reflected back through the EOM. This is achieved using a 50:50 beam splitter (BS) which partially transmits light to the beam dump and partially reflects light. Thereafter the light is reflected by mirror M1 and passes back through the EOM where 50 $\%$ of the light is then transmitted towards Bob. 

Another 50:50 BS cube is used to split the light across two path arms. When a single photon interacts with the BS, the output is entangled with the vacuum input where the expectation value of either output is 1/2. Therefore, the Bob's measurement basis is chosen at random. Each arm contains a polarising beam splitter (PBS) followed by two avalanche photodiodes (APDs) acting as single-photon counters. The polarising beam splitter transmits linearly polarised light of one state and reflects the opposite state. The addition of a quarter-wave plate on arm 1 transforms the polarisation state by a phase retardation of $\Delta \phi = \pi/2$. Therefore introducing a 45$^{\circ}$ angle between incoming light polarisation and the optical axis transforms the states as: $\ket{L}\leftrightarrow\ket{V}$ and $\ket{R}\leftrightarrow\ket{H}$. If the light is initially circularly polarised using this optical arrangement only one detector on arm 2 will detect a photon event. The detection events are processed using a time-digital converter which time-bins the arrival photon events for each channel, in addition to recording the trigger pulse. 
